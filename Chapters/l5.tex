\chapter*{Macro Lecture II.5: The AD-AS Model}
\section*{Overview}
This lecture introduces the AD-AS framework, which integrates the goods, money, and labor markets to analyze how aggregate demand and aggregate supply determine both output and the price level. The model accommodates adjustments to shocks in both AD and AS.

\section*{Aggregate Demand (AD)}
\begin{itemize}
    \item AD is the total spending on final goods and services and is given by:
    \[
    AD = C + I + G + (X - Z)
    \]
    \item Monetary policy influences AD through its impact on consumption and investment.
    \item Higher inflation typically leads the Central Bank to raise real interest rates, reducing consumption and investment, thereby shifting AD leftward.
\end{itemize}

\section*{Aggregate Supply (AS)}
\begin{itemize}
    \item \textbf{Long-Run Aggregate Supply (LRAS):} With fully flexible wages and prices, the economy produces at its potential output:
    \[
    Y = Y^*
    \]
    \begin{itemize}
        \item \textbf{Potential Output:} Defined as the level of output achieved when every market in the economy is in long-run equilibrium and all economic agents correctly anticipate economic conditions. It represents the sustainable level of production, not the maximum possible output.
    \end{itemize}
    \item \textbf{Short-Run Aggregate Supply (SRAS):} Due to sticky wages, SRAS is upward sloping. Firms adjust production based on the real wage:
    \[
    MP_L = \frac{w}{p}
    \]
    \item \textbf{Shifts in SRAS:}
    \begin{itemize}
        \item When output is below potential, higher unemployment may lead to lower nominal wages, shifting SRAS rightward.
        \item When output exceeds potential, rising wage demands shift SRAS leftward.
    \end{itemize}
\end{itemize}

\section*{Adjustment Mechanisms}
\begin{itemize}
    \item \textbf{AD Shocks:} Changes in fiscal or monetary policy can shift AD. The Central Bank may adjust interest rates to counteract inflationary or deflationary pressures.
    \item \textbf{AS Shocks:} Temporary supply shocks affect SRAS without altering potential output, whereas permanent shocks can shift both SRAS and LRAS.
\end{itemize}

\section*{The Taylor Rule and Monetary Policy}
\begin{itemize}
    \item The Taylor rule guides Central Bank policy by adjusting the nominal interest rate based on deviations of inflation and output from their targets:
    \[
    i - i^* = (1+a)(\pi - \pi^*) + b(Y - Y^*)
    \]
    \item This rule helps stabilize inflation while influencing aggregate demand.
\end{itemize}

\end{document}