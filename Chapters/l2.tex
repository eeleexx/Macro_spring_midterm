\chapter*{Macro Lecture II.2: Output and Aggregate Demand -- The Keynesian Cross Model}
\section*{Overview}
This lecture presents the basic short-run goods-and-services market model, emphasizing the Keynesian cross framework. It covers equilibrium in the goods market, the multiplier effect, the paradox of thrift, and the role of government in aggregate demand.

\section*{The Basic Model without Government}
\begin{itemize}
    \item \textbf{Consumption Function:}
    \[
    C = A + cY
    \]
    where \(A\) is autonomous consumption and \(c\) (with \(0 < c < 1\)) is the marginal propensity to consume (MPC).
    \item \textbf{Investment:}
    \[
    I = \bar{I}
    \]
    (investment is treated as exogenous).
    \item \textbf{Aggregate Demand (AD):}
    \[
    AD = A + cY + \bar{I}
    \]
    \item \textbf{Equilibrium Condition:}
    \[
    Y = AD = A + cY + \bar{I}
    \]
    Solving for equilibrium output:
    \[
    Y^* = \frac{A + \bar{I}}{1 - c}
    \]
    Here, \(\frac{1}{1-c}\) is the multiplier.
\end{itemize}

\section*{Market Adjustments and Disequilibria}
\begin{itemize}
    \item When actual output \(Y\) is below \(Y^*\), aggregate demand exceeds output, leading to unplanned reductions in inventories and signaling firms to increase production.
    \item Conversely, when \(Y\) exceeds \(Y^*\), firms face unsold inventories, prompting a reduction in output.
\end{itemize}

\section*{Investment Equals Saving}
In equilibrium, the goods market satisfies:
\[
Y = C + I \quad \Rightarrow \quad S = Y - C = I
\]

\section*{Extension: Adding Government}
\begin{itemize}
    \item With government, disposable income is:
    \[
    Y_d = (1-t)Y
    \]
    where \(t\) is the net tax rate.
    \item The consumption function becomes:
    \[
    C = A + c(1-t)Y
    \]
    \item The equilibrium condition now is:
    \[
    Y = A + c(1-t)Y + \bar{I} + G
    \]
    Solving for \(Y^*\):
    \[
    Y^* = \frac{A + \bar{I} + G}{1 - c(1-t)}
    \]
\end{itemize}

\section*{Balanced Budget Multiplier}
\begin{itemize}
    \item If government spending \(G\) increases by 1 dollar and is financed by an equal increase in taxes, the equilibrium output will increase. Under certain conditions, the balanced budget multiplier is equal to 1.
\end{itemize}

\section*{Additional Discussion}
\begin{itemize}
    \item \textbf{Actual vs. Potential Output:} While potential output is fixed in the short run, actual output can deviate, leading to economic recessions or expansions.
    \item \textbf{Paradox of Thrift:} An increase in household savings (modeled as a decrease in \(A\)) may reduce aggregate demand and lower equilibrium output, leaving overall private saving unchanged but reducing total income.
\end{itemize}