\chapter*{Macro Lecture II.1: Introduction to Macroeconomics, GDP, and the Circular Flow Model}
\section*{Overview}
This lecture introduces macroeconomics at the aggregate level, focusing on the measurement of national output and income through GDP, and explains the circular flow model that illustrates the interactions among households, firms, government, and the foreign sector.

\section*{Key Concepts in Macroeconomics}
\begin{itemize}
    \item \textbf{Aggregate Analysis:} Examines the economy as a whole rather than individual markets.
    \item \textbf{General Equilibrium:} Considers the simultaneous equilibrium in goods, factor, and financial markets.
    \item \textbf{Core Issues:} Economic growth, inflation, unemployment, business cycles, and economic crises.
\end{itemize}

\section*{GDP and National Income Statistics}
\begin{itemize}
    \item \textbf{Definition of GDP:} The market value of final goods and services produced within a country over a given period.
    \item \textbf{Methods of Computing GDP:}
    \begin{itemize}
        \item \textbf{Output (Value Added) Approach:} Summing the value added at each stage of production.
        \item \textbf{Expenditure Approach:} Adding up consumption (C), investment (I), government spending (G), and net exports (exports $X$ minus imports $Z$):
        \[
        GDP = C + I + G + (X - Z)
        \]
        \item \textbf{Income Approach:} Summing incomes earned by households and firms (wages, profits, etc.).
    \end{itemize}
    \item \textbf{Nominal vs. Real GDP:}
    \begin{itemize}
        \item \textbf{Nominal GDP:} Calculated at current market prices.
        \item \textbf{Real GDP:} Calculated at constant prices to adjust for inflation.
    \end{itemize}
    \item \textbf{GDP Deflator:} A measure of the price level, defined as the ratio of Nominal GDP to Real GDP.
\end{itemize}

\section*{The Circular Flow Model}
\begin{itemize}
    \item \textbf{Basic Framework:} Describes how money flows between households and firms through the exchange of goods, services, and factors of production.
    \item \textbf{Leakages and Injections:}
    \begin{itemize}
        \item \textbf{Leakages:} Portions of income not spent on domestic goods and services, such as savings, taxes, and imports.
        \item \textbf{Injections:} Additional spending into the economy, such as investment, government spending, and exports.
    \end{itemize}
    \item \textbf{Savings and Investment Identity:}
    \[
    Y = C + S \quad \text{and} \quad Y = C + I \quad \Rightarrow \quad S = I
    \]
    \item \textbf{Role of Government and Foreign Sector:} Government activities (taxes and spending) and foreign trade (exports and imports) further modify the flow of income.
\end{itemize}

\section*{Additional Discussion Points}
\begin{itemize}
    \item The importance of using market values and focusing on final goods to avoid double counting.
    \item How GDP per capita is used as an indicator of average living standards.
    \item The implications of including unsold inventory as investment in GDP.
\end{itemize}