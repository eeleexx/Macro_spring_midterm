\chapter*{Macro Lecture II.3: The Keynesian Cross Model, Part 2 -- Money and Banking}
\section*{Overview}
This lecture extends the Keynesian cross framework by incorporating topics on fiscal policy, the foreign sector, and, in depth, money and banking. It discusses how monetary policy and banking operations interact with aggregate demand and influence the economy.

\section*{Fiscal Policy, Budget, and the Foreign Sector}
\begin{itemize}
    \item \textbf{Fiscal Policy and Fiscal Stance:} 
    \begin{itemize}
        \item Fiscal policy is the government’s use of spending and taxation to influence aggregate demand.
        \item The \emph{fiscal stance} indicates whether the policy is expansionary (budget deficit) or contractionary (budget surplus).
        \item Adjustments such as structural budgets and inflation-adjusted budgets are used to assess the true fiscal stance.
    \end{itemize}
    \item \textbf{Budget Deficit and Debt:}
    \begin{itemize}
        \item Governments often finance deficits via debt. A key indicator is the debt-to-GDP ratio.
        \item Strategies for coping with debt include promoting nominal GDP growth, using inflation to reduce real debt burdens, or, in extreme cases, default.
    \end{itemize}
    \item \textbf{Adding the Foreign Sector:}
    \begin{itemize}
        \item Net exports (NX) are defined as \(NX = X - Z\), where exports \(X\) are typically treated as exogenous and imports \(Z\) are modeled as \(Z = zY\).
        \item The equilibrium condition is then extended to:
        \[
        Y = C + I + G + X - zY
        \]
    \end{itemize}
\end{itemize}

\section*{Money and Banking}
\begin{itemize}
    \item \textbf{Definitions and Money Aggregates:}
    \begin{itemize}
        \item \emph{M0} is the currency outside banks.
        \item \emph{MB} (monetary base) equals cash plus bank reserves.
        \item \emph{M1} includes cash and demand deposits.
    \end{itemize}
    \item \textbf{Banking Operations and Money Creation:}
    \begin{itemize}
        \item A commercial bank’s balance sheet satisfies:
        \[
        \text{Assets} = \text{Liabilities} + \text{Equity}
        \]
        \item Money creation occurs through the deposit and loan multipliers:
        \[
        \text{Deposit Multiplier} = \frac{1}{rr}, \quad \text{Loan Multiplier} = \frac{1 - rr}{rr}
        \]
    \end{itemize}
    \item \textbf{Money Multiplier:}
    \[
    mm = \frac{1 + cr}{cr + rr + er}
    \]
    where the ratios \(cr\), \(rr\), and \(er\) capture public preferences and banking regulations.
    \item \textbf{Money Market Equilibrium:}
    \begin{itemize}
        \item Equilibrium in the money market is reached when money demand \(M_d(Y, r)\) equals money supply \(M_s\).
        \item Real money balances are expressed as \(\frac{M}{P}\).
    \end{itemize}
\end{itemize}

\section*{Monetary Policy and Its Instruments}
\begin{itemize}
    \item \textbf{Objectives:} The Central Bank conducts monetary policy to stabilize output and promote economic growth.
    \item \textbf{Instruments:}
    \begin{itemize}
        \item \emph{Required Reserve Ratio:} Adjusting this influences the amount of money banks can create.
        \item \emph{Open Market Operations:} Buying or selling government bonds to change the monetary base.
        \item \emph{Discount Rate:} Changing the rate at which commercial banks borrow from the Central Bank affects bank reserves and lending.
    \end{itemize}
    \item \textbf{Transmission Mechanism:}
    \begin{itemize}
        \item Lower interest rates stimulate investment and consumption.
        \item The overall effect of monetary policy is transmitted through changes in aggregate demand.
    \end{itemize}
\end{itemize}