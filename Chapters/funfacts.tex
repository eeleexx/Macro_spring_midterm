\chapter*{Last-minute additions after the consultation}

\section*{Prerequisites: Keynesian, IS-LM, and AD-AS Frameworks}
A solid understanding of the following models is essential. 

\subsection*{Comparison of Macroeconomic Frameworks}

\begin{table}[ht]
    \centering
    \small
    \begin{tabularx}{\textwidth}{|X|X|X|X|}
    \hline
    \textbf{Model} & \textbf{Price and Wage Flexibility} & \textbf{Core Mechanism} & \textbf{Policy Implications} \\
    \hline
    Keynesian Cross & Fixed prices and wages & Demand determines output; government spending has a direct effect & Fiscal policy is highly effective; monetary policy not considered \\
    \hline
    IS-LM Model & Fixed prices in short run & Interaction of goods (IS) and money (LM) markets determines interest rates and output & Both fiscal and monetary policies matter; crowding-out effect can occur \\
    \hline
    AD-AS Model & Short-run price stickiness; long-run flexibility & Aggregate demand and supply interaction determines output and prices & Long-run effects depend on supply-side factors; both demand and supply policies matter \\
    \hline
    \end{tabularx}
    \caption{Comparison of Keynesian, IS-LM, and AD-AS Models}
\end{table}

\newpage

The formulas below are provided for reference of what are the factors affecting slopes, not some actual formulas which should be consulted while solving tasks.

\textbf{Keynesian Cross:}
\begin{align*}
    Y &= C + I + G \\
    C &= C_0 + c(Y - T) \\
    \text{Multiplier} &= \frac{1}{1 - c (1 - t) + m}
\end{align*}

\textbf{IS-LM Model:}
\begin{align*}
    \text{IS: } & Y = C(Y - T) + I(r) + G \\
    \text{LM: } & \frac{M}{P} = L(Y, r)
\end{align*}

\textbf{AD-AS Model:}
\begin{align*}
    \text{AD: } & Y = f(P, \text{demand}) \\
    \text{SRAS: } & Y = f(W, \text{costs}) \\
    \text{LRAS: } & Y = Y^*
\end{align*}

\section*{Money and Banking Multipliers}
\begin{itemize}
    \item \textbf{Money Multiplier:} When the \emph{required reserve ratio} falls, the money multiplier increases. This expansion in the multiplier effect enlarges the monetary base, subsequently shifting the Aggregate Demand (AD) curve to the right.
    \item \textbf{Banking Multiplier:} Similarly, the ability of banks to create loans (the banking multiplier) is enhanced when reserve requirements are relaxed, further contributing to an increase in the money supply.
\end{itemize}

\section*{Interactions among IS, LM, and AD Curves}
\begin{itemize}
    \item The AD curve is closely linked to the IS-LM framework. A rightward shift in the IS curve—indicating increased equilibrium output—typically results in a corresponding rightward shift in AD.
    \item \textbf{Expansionary Policies:} Stimulative measures, such as fiscal or monetary expansion, shift the IS, LM, and AD curves to the right, reinforcing aggregate demand across multiple channels.
\end{itemize}

\section*{Supply-Side Adjustments}
\begin{itemize}
    \item \textbf{Short-Run Aggregate Supply (SRAS):} The SRAS curve is primarily influenced by changes in the labor market. For example, shifts in wage dynamics directly affect production costs and, consequently, the supply side of the economy.
    \item If nominal wages increase—and if real wages rise accordingly—production costs go up. Firms may then hire fewer workers, leading to a leftward shift in the SRAS curve.
\end{itemize}

\section*{Marginal Propensity to Consume (MPC) and Its Impact on AD}
\begin{itemize}
    \item An increase in the \emph{marginal propensity to consume} (MPC) raises overall consumption (C), which is a major component of aggregate demand.
    \item In general, any factor that boosts consumption or investment will shift the AD curve to the right.
\end{itemize}