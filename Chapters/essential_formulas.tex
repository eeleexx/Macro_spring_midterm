\chapter*{Essential Formulas}

\subsection*{Macroeconomic GDP Formulas}
\begin{itemize}
    \item Expenditure Approach:
    \[ GDP = C + I + G + (X - Z) \]
    \item Income Approach:
    \[ GDP = \text{Labor Income} + \text{Capital Income} \]
    \item Output (Value Added) Approach:
    \[ GDP = \sum \text{Value Added} \]
    \item Savings-Investment Identity:
    \[ Y = C + S \quad \text{and} \quad Y = C + I \quad \Rightarrow \quad S = I \]
    \item Nominal GDP:
    \[ \text{Nominal GDP} = \sum (\text{Quantity} \times \text{Current Price}) \]
    \item Real GDP:
    \[ \text{Real GDP} = \sum (\text{Quantity} \times \text{Constant Price}) \]
    \item GDP Deflator:
    \[ \text{GDP Deflator} = \frac{\text{Nominal GDP}}{\text{Real GDP}} \times 100 \]
\end{itemize}

\subsection*{Keynesian Cross Model Formulas}
\begin{itemize}
    \item \textbf{Consumption Function (No Government):}
    \[
    C = A + cY
    \]
    where \(A\) is autonomous consumption and \(c\) (with \(0<c<1\)) is the marginal propensity to consume.
    \item \textbf{Investment (Exogenous):}
    \[
    I = \bar{I}
    \]
    \item \textbf{Aggregate Demand (AD):}
    \[
    AD = A + cY + \bar{I}
    \]
    \item \textbf{Equilibrium Output (No Government):}
    \[
    Y^* = \frac{A + \bar{I}}{1 - c}
    \]
    Here, \(\frac{1}{1-c}\) is the multiplier.
    \item \textbf{Consumption Function (With Government):}
    \[
    C = A + c(1-t)Y
    \]
    where \(t\) is the net tax rate.
    \item \textbf{Equilibrium Output (With Government):}
    \[
    Y^* = \frac{A + \bar{I} + G}{1 - c(1-t)}
    \]
    \item \textbf{Savings-Investment Identity:}
    \[
    S = Y - C = I
    \]
    \item \textbf{Balanced Budget Multiplier:} Conceptually, a \$1 increase in government spending financed by an equal increase in taxes leads to an increase in equilibrium output by \$1.
\end{itemize}

\subsection*{Additional Macroeconomic Formulas (Lecture II.3)}
\begin{itemize}
    \item \textbf{Open Economy Equilibrium (with Government):}
    \[
    Y^* = \frac{A + \bar{I} + G + \bar{X}}{1 - c(1-t) + z}
    \]
    where \(z\) is the marginal propensity to import.
    \item \textbf{Real Interest Rate (Inflation-Adjusted):}
    \[
    r_{\text{real}} \approx r_{\text{nominal}} - \pi
    \]
    with \(\pi\) being the inflation rate.
    \item \textbf{Debt-to-GDP Ratio:}
    \[
    \text{Debt-to-GDP Ratio} = \frac{\text{Government Debt}}{Y}
    \]
    \item \textbf{Money Multiplier:}
    \[
    mm = \frac{1 + cr}{cr + rr + er}
    \]
    where \(cr\) is the currency-deposit ratio, \(rr\) the required reserve ratio, and \(er\) the excess reserve ratio.
    \item \textbf{Deposit Multiplier:}
    \[
    \text{Deposit Multiplier} = \frac{1}{rr}
    \]
    \item \textbf{Loan Multiplier:}
    \[
    \text{Loan Multiplier} = \frac{1 - rr}{rr}
    \]
    \item \textbf{Money Market Equilibrium:}
    \[
    M_d(Y, r) = M_s
    \]
    and real money balances are given by \(\frac{M}{P}\).
    \item \textbf{Definitions of Money Aggregates:}
    \[
    M0 = \text{Currency outside banks}
    \]
    \[
    MB = M0 + \text{Reserves held by banks}
    \]
    \[
    M1 = C + D \quad (\text{with } C=\text{currency},\, D=\text{demand deposits})
    \]
\end{itemize}

\subsection*{IS-MP Model Formulas (Lecture II.4)}
\begin{itemize}
    \item \textbf{Investment Function:}
    \[
    I = I_0 + \text{mpi} \cdot Y - b \cdot i
    \]
    where \(I_0 > 0\) is autonomous investment, \(\text{mpi} \in (0,1)\) is the marginal propensity to invest, and \(b > 0\) captures investment’s sensitivity to the interest rate.
    \item \textbf{Consumption Function:}
    \[
    C = A_0 - a \cdot i + c(1-t) \cdot Y
    \]
    where \(A_0 > 0\) is autonomous consumption, \(a > 0\) is the sensitivity of consumption to the interest rate, \(c \in (0,1)\) is the marginal propensity to consume, and \(t\) is the net tax rate.
    \item \textbf{Goods Market Equilibrium:}
    \[
    Y = C + I + G + NX
    \]
    with \(NX\) representing net exports.
    \item \textbf{Derived IS Curve:}
    \[
    Y\Bigl(1 - c(1-t) - \text{mpi} + z\Bigr) = \Bigl[A_0 + I_0 + G + X\Bigr] - (a+b)i
    \]
    where \(z\) is the marginal propensity to import.
    \item \textbf{Equilibrium Output (IS Relation):}
    \[
    Y^* = \frac{A_0 + I_0 + G + X - (a+b)i}{1 - c(1-t) - \text{mpi} + z}
    \]
    \item \textbf{Monetary Policy Assumption:} Under fixed prices (GDP deflator \(=1\)), nominal and real interest rates are identical.
\end{itemize}

\subsection*{AD-AS Model and Taylor Rule (Lecture II.5)}
\begin{itemize}
    \item \textbf{Aggregate Demand (AD):}
    \[
    AD = C + I + G + (X - Z)
    \]
    (Note: In this model, AD is influenced by monetary policy through its effect on consumption and investment.)
    \item \textbf{Short-Run Aggregate Supply (SRAS):}
    Due to sticky wages, the SRAS is upward sloping. Firms determine output based on the real wage:
    \[
    MPL = \frac{w}{p}
    \]
    where \(MPL\) is the marginal product of labor, \(w\) the nominal wage, and \(p\) the price level.
    \item \textbf{Long-Run Aggregate Supply (LRAS):}
    With flexible wages and prices, the economy operates at potential output:
    \[
    Y = Y^*
    \]
    \item \textbf{The Taylor Rule:}
    The rule for setting the nominal interest rate is given by:
    \[
    r - r^* = a(\pi - \pi^*) + b(Y - Y^*)
    \]
    or, for nominal rates,
    \[
    i - i^* = (1+a)(\pi - \pi^*) + b(Y - Y^*)
    \]
    where \(a>0\) and \(b>0\) measure the responsiveness to inflation and the output gap.
\end{itemize}